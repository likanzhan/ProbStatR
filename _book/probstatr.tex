\documentclass[]{ctexbook}
\usepackage{lmodern}
\usepackage{amssymb,amsmath}
\usepackage{ifxetex,ifluatex}
\usepackage{fixltx2e} % provides \textsubscript
\ifnum 0\ifxetex 1\fi\ifluatex 1\fi=0 % if pdftex
  \usepackage[T1]{fontenc}
  \usepackage[utf8]{inputenc}
\else % if luatex or xelatex
  \ifxetex
    \usepackage{xltxtra,xunicode}
  \else
    \usepackage{fontspec}
  \fi
  \defaultfontfeatures{Ligatures=TeX,Scale=MatchLowercase}
\fi
% use upquote if available, for straight quotes in verbatim environments
\IfFileExists{upquote.sty}{\usepackage{upquote}}{}
% use microtype if available
\IfFileExists{microtype.sty}{%
\usepackage{microtype}
\UseMicrotypeSet[protrusion]{basicmath} % disable protrusion for tt fonts
}{}
\usepackage[b5paper,tmargin=2.5cm,bmargin=2.5cm,lmargin=3.5cm,rmargin=2.5cm]{geometry}
\usepackage[unicode=true]{hyperref}
\PassOptionsToPackage{usenames,dvipsnames}{color} % color is loaded by hyperref
\hypersetup{
            pdftitle={R语言和统计},
            pdfauthor={战立侃},
            colorlinks=true,
            linkcolor=Maroon,
            citecolor=Blue,
            urlcolor=Blue,
            breaklinks=true}
\urlstyle{same}  % don't use monospace font for urls
\usepackage{natbib}
\bibliographystyle{apalike}
\usepackage{color}
\usepackage{fancyvrb}
\newcommand{\VerbBar}{|}
\newcommand{\VERB}{\Verb[commandchars=\\\{\}]}
\DefineVerbatimEnvironment{Highlighting}{Verbatim}{commandchars=\\\{\}}
% Add ',fontsize=\small' for more characters per line
\usepackage{framed}
\definecolor{shadecolor}{RGB}{248,248,248}
\newenvironment{Shaded}{\begin{snugshade}}{\end{snugshade}}
\newcommand{\AlertTok}[1]{\textcolor[rgb]{0.94,0.16,0.16}{#1}}
\newcommand{\AnnotationTok}[1]{\textcolor[rgb]{0.56,0.35,0.01}{\textbf{\textit{#1}}}}
\newcommand{\AttributeTok}[1]{\textcolor[rgb]{0.77,0.63,0.00}{#1}}
\newcommand{\BaseNTok}[1]{\textcolor[rgb]{0.00,0.00,0.81}{#1}}
\newcommand{\BuiltInTok}[1]{#1}
\newcommand{\CharTok}[1]{\textcolor[rgb]{0.31,0.60,0.02}{#1}}
\newcommand{\CommentTok}[1]{\textcolor[rgb]{0.56,0.35,0.01}{\textit{#1}}}
\newcommand{\CommentVarTok}[1]{\textcolor[rgb]{0.56,0.35,0.01}{\textbf{\textit{#1}}}}
\newcommand{\ConstantTok}[1]{\textcolor[rgb]{0.00,0.00,0.00}{#1}}
\newcommand{\ControlFlowTok}[1]{\textcolor[rgb]{0.13,0.29,0.53}{\textbf{#1}}}
\newcommand{\DataTypeTok}[1]{\textcolor[rgb]{0.13,0.29,0.53}{#1}}
\newcommand{\DecValTok}[1]{\textcolor[rgb]{0.00,0.00,0.81}{#1}}
\newcommand{\DocumentationTok}[1]{\textcolor[rgb]{0.56,0.35,0.01}{\textbf{\textit{#1}}}}
\newcommand{\ErrorTok}[1]{\textcolor[rgb]{0.64,0.00,0.00}{\textbf{#1}}}
\newcommand{\ExtensionTok}[1]{#1}
\newcommand{\FloatTok}[1]{\textcolor[rgb]{0.00,0.00,0.81}{#1}}
\newcommand{\FunctionTok}[1]{\textcolor[rgb]{0.00,0.00,0.00}{#1}}
\newcommand{\ImportTok}[1]{#1}
\newcommand{\InformationTok}[1]{\textcolor[rgb]{0.56,0.35,0.01}{\textbf{\textit{#1}}}}
\newcommand{\KeywordTok}[1]{\textcolor[rgb]{0.13,0.29,0.53}{\textbf{#1}}}
\newcommand{\NormalTok}[1]{#1}
\newcommand{\OperatorTok}[1]{\textcolor[rgb]{0.81,0.36,0.00}{\textbf{#1}}}
\newcommand{\OtherTok}[1]{\textcolor[rgb]{0.56,0.35,0.01}{#1}}
\newcommand{\PreprocessorTok}[1]{\textcolor[rgb]{0.56,0.35,0.01}{\textit{#1}}}
\newcommand{\RegionMarkerTok}[1]{#1}
\newcommand{\SpecialCharTok}[1]{\textcolor[rgb]{0.00,0.00,0.00}{#1}}
\newcommand{\SpecialStringTok}[1]{\textcolor[rgb]{0.31,0.60,0.02}{#1}}
\newcommand{\StringTok}[1]{\textcolor[rgb]{0.31,0.60,0.02}{#1}}
\newcommand{\VariableTok}[1]{\textcolor[rgb]{0.00,0.00,0.00}{#1}}
\newcommand{\VerbatimStringTok}[1]{\textcolor[rgb]{0.31,0.60,0.02}{#1}}
\newcommand{\WarningTok}[1]{\textcolor[rgb]{0.56,0.35,0.01}{\textbf{\textit{#1}}}}
\usepackage{longtable,booktabs}
% Fix footnotes in tables (requires footnote package)
\IfFileExists{footnote.sty}{\usepackage{footnote}\makesavenoteenv{long table}}{}
\usepackage{graphicx,grffile}
\makeatletter
\def\maxwidth{\ifdim\Gin@nat@width>\linewidth\linewidth\else\Gin@nat@width\fi}
\def\maxheight{\ifdim\Gin@nat@height>\textheight\textheight\else\Gin@nat@height\fi}
\makeatother
% Scale images if necessary, so that they will not overflow the page
% margins by default, and it is still possible to overwrite the defaults
% using explicit options in \includegraphics[width, height, ...]{}
\setkeys{Gin}{width=\maxwidth,height=\maxheight,keepaspectratio}
\IfFileExists{parskip.sty}{%
\usepackage{parskip}
}{% else
\setlength{\parindent}{0pt}
\setlength{\parskip}{6pt plus 2pt minus 1pt}
}
\setlength{\emergencystretch}{3em}  % prevent overfull lines
\providecommand{\tightlist}{%
  \setlength{\itemsep}{0pt}\setlength{\parskip}{0pt}}
\setcounter{secnumdepth}{5}
% Redefines (sub)paragraphs to behave more like sections
\ifx\paragraph\undefined\else
\let\oldparagraph\paragraph
\renewcommand{\paragraph}[1]{\oldparagraph{#1}\mbox{}}
\fi
\ifx\subparagraph\undefined\else
\let\oldsubparagraph\subparagraph
\renewcommand{\subparagraph}[1]{\oldsubparagraph{#1}\mbox{}}
\fi

% set default figure placement to htbp
\makeatletter
\def\fps@figure{htbp}
\makeatother

\usepackage{ctex} % By likan
\usepackage{booktabs}
\usepackage{longtable}

\usepackage{framed,color}
\definecolor{shadecolor}{RGB}{248,248,248}

\renewcommand{\textfraction}{0.05}
\renewcommand{\topfraction}{0.8}
\renewcommand{\bottomfraction}{0.8}
\renewcommand{\floatpagefraction}{0.75}

\let\oldhref\href
\renewcommand{\href}[2]{#2\footnote{\url{#1}}}

\makeatletter
\newenvironment{kframe}{%
\medskip{}
\setlength{\fboxsep}{.8em}
 \def\at@end@of@kframe{}%
 \ifinner\ifhmode%
  \def\at@end@of@kframe{\end{minipage}}%
  \begin{minipage}{\columnwidth}%
 \fi\fi%
 \def\FrameCommand##1{\hskip\@totalleftmargin \hskip-\fboxsep
 \colorbox{shadecolor}{##1}\hskip-\fboxsep
     % There is no \\@totalrightmargin, so:
     \hskip-\linewidth \hskip-\@totalleftmargin \hskip\columnwidth}%
 \MakeFramed {\advance\hsize-\width
   \@totalleftmargin\z@ \linewidth\hsize
   \@setminipage}}%
 {\par\unskip\endMakeFramed%
 \at@end@of@kframe}
\makeatother

\makeatletter
\@ifundefined{Shaded}{
}{\renewenvironment{Shaded}{\begin{kframe}}{\end{kframe}}}
\makeatother
% https://github.com/CTeX-org/ctex-kit/issues/331
\RecustomVerbatimEnvironment{Highlighting}{Verbatim}{commandchars=\\\{\},formatcom=\xeCJKVerbAddon}

\usepackage{makeidx}
\makeindex

\urlstyle{tt}

\usepackage{amsthm}
\makeatletter
\def\thm@space@setup{%
  \thm@preskip=8pt plus 2pt minus 4pt
  \thm@postskip=\thm@preskip
}
\makeatother

\frontmatter

\title{R语言和统计}
\author{战立侃}
\date{2018-05-28}

\let\BeginKnitrBlock\begin \let\EndKnitrBlock\end
\begin{document}
\maketitle


\thispagestyle{empty}

\begin{center}
谨以此书献给大巫、小巫、和阿毛。
\end{center}

\setlength{\abovedisplayskip}{-5pt}
\setlength{\abovedisplayshortskip}{-5pt}

{
\setcounter{tocdepth}{2}
\tableofcontents
}
\listoftables
\listoffigures
\chapter*{前言}


2018年,我立了一个大大的FLAG。我准备写一本书。这本书是关于R语言和统计学的。但是具体写什么,我还没有想好。例如,第
\ref{Introduction-to-Statistics} 章介绍了什么是统计学;第 \ref{wind}
章介绍了描述统计中的集中量数和离散量数。

本书的主要参考书有:行为科学统计学
\citep{RN154}、用R学习概率论和数理统计
\citep{RN261}、应用回归分析和一般化线性模型 \citep{RN146}。

\section*{致谢}


非常感谢我的父母,我家的大巫、小巫、阿毛的一如继往的支持。

\BeginKnitrBlock{flushright}
战立侃 2018年于北京
\EndKnitrBlock{flushright}

\hypertarget{author.unnumbered}{%
\chapter{关于作者}\label{author.unnumbered}}

我是谁?

\mainmatter

\hypertarget{Introduction-to-Statistics}{%
\chapter{统计学简介}\label{Introduction-to-Statistics}}

\section{统计学、科学、和观察}

\begin{itemize}
\item
  统计学 (Statistics) 指一系列用于组织、总结、和理解信息的数学过程。
\item
  通过组织和总结信息,研究者能够更好的理解和交流研究结果。
\end{itemize}

\section{数据结构、研究方法、和统计学}

\section{变量和测量}

\section{统计符号}

\mainmatter

\hypertarget{Plot}{%
\chapter{统计作图}\label{Plot}}

\begin{center}\includegraphics{probstatr_files/figure-latex/Figure_03_01-1} \end{center}

\hypertarget{central-tendency}{%
\chapter{描述统计}\label{central-tendency}}

描述数据的两个基本指标是集中趋势和离散趋势。

\section{集中趋势}

话说张老爷子写了一首诗:

\begin{quote}
姑苏开遍碧桃时,邂逅河阳女画师。\\
红豆江南留梦影,白苹风末唱秋词。
\end{quote}

\section{离散趋势}

貌似大家都喜欢用白萍风这个意境。又如彭玉麟的对联:

\begin{quote}
凭栏看云影波光,最好是红蓼花疏、白苹秋老;\\
把酒对琼楼玉宇,莫孤负天心月到、水面风来。
\end{quote}

嘿,玛尼玛尼哄。

\cleardoublepage

\hypertarget{appendix-}{%
\appendix \addcontentsline{toc}{chapter}{\appendixname}}


\hypertarget{sound}{%
\chapter{后记}\label{sound}}

To be added

\hypertarget{colors}{%
\chapter{常用颜色}\label{colors}}

\begin{enumerate}
\def\labelenumi{\arabic{enumi}.}
\setcounter{enumi}{1}
\tightlist
\item
  blue:\texttt{\#005093}、\texttt{\#0bb0ae}
\end{enumerate}

\begin{Shaded}
\begin{Highlighting}[]
\NormalTok{stbc <-}\StringTok{ }\ControlFlowTok{function}\NormalTok{ (x)\{}
  \KeywordTok{hist}\NormalTok{(mtcars}\OperatorTok{$}\NormalTok{mpg, }\DataTypeTok{main =} \StringTok{""}\NormalTok{, }\DataTypeTok{xlab =} \StringTok{"MPG"}\NormalTok{, }\DataTypeTok{col =}\NormalTok{ x)}
\NormalTok{\}}
\end{Highlighting}
\end{Shaded}

\begin{figure}

{\centering \includegraphics[width=0.7\linewidth]{probstatr_files/figure-latex/hex005093-1} 

}

\caption{005093}\label{fig:hex005093}
\end{figure}

\begin{figure}

{\centering \includegraphics[width=0.7\linewidth]{probstatr_files/figure-latex/hex0bb0ae-1} 

}

\caption{00bb0ae}\label{fig:hex0bb0ae}
\end{figure}

\begin{enumerate}
\def\labelenumi{\arabic{enumi}.}
\setcounter{enumi}{2}
\tightlist
\item
  brown: col: \texttt{\#fce6bf}; border: \texttt{\#b78135},
  \texttt{\#ef9d56}
\end{enumerate}

\begin{Shaded}
\begin{Highlighting}[]
\KeywordTok{stbc}\NormalTok{(}\StringTok{"#fce6bf"}\NormalTok{)}
\end{Highlighting}
\end{Shaded}

\includegraphics{probstatr_files/figure-latex/unnamed-chunk-4-1.pdf}

\begin{Shaded}
\begin{Highlighting}[]
\KeywordTok{stbc}\NormalTok{(}\StringTok{"#b78135"}\NormalTok{)}
\end{Highlighting}
\end{Shaded}

\includegraphics{probstatr_files/figure-latex/unnamed-chunk-4-2.pdf}

\begin{Shaded}
\begin{Highlighting}[]
\KeywordTok{stbc}\NormalTok{(}\StringTok{"#ef9d56"}\NormalTok{)}
\end{Highlighting}
\end{Shaded}

\includegraphics{probstatr_files/figure-latex/unnamed-chunk-4-3.pdf} 4.
green: \texttt{\#00acaa}, \texttt{\#2aa96d}

\begin{Shaded}
\begin{Highlighting}[]
\KeywordTok{stbc}\NormalTok{(}\StringTok{"#00acaa"}\NormalTok{)}
\end{Highlighting}
\end{Shaded}

\includegraphics{probstatr_files/figure-latex/unnamed-chunk-5-1.pdf}

\begin{Shaded}
\begin{Highlighting}[]
\KeywordTok{stbc}\NormalTok{(}\StringTok{"#2aa96d"}\NormalTok{)}
\end{Highlighting}
\end{Shaded}

\includegraphics{probstatr_files/figure-latex/unnamed-chunk-5-2.pdf}

\begin{enumerate}
\def\labelenumi{\arabic{enumi}.}
\setcounter{enumi}{4}
\tightlist
\item
  red: \texttt{\#cf232a}, \texttt{\#d5493a}
\end{enumerate}

\begin{Shaded}
\begin{Highlighting}[]
\KeywordTok{stbc}\NormalTok{(}\StringTok{"#cf232a"}\NormalTok{)}
\end{Highlighting}
\end{Shaded}

\includegraphics{probstatr_files/figure-latex/unnamed-chunk-6-1.pdf}

\begin{Shaded}
\begin{Highlighting}[]
\KeywordTok{stbc}\NormalTok{(}\StringTok{"#d5493a"}\NormalTok{)}
\end{Highlighting}
\end{Shaded}

\includegraphics{probstatr_files/figure-latex/unnamed-chunk-6-2.pdf}

\bibliography{book.bib,packages.bib,Statistics.bib}

\backmatter
\printindex

\end{document}
